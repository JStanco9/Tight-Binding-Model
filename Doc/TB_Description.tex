\documentclass{article}
\usepackage{amsmath}
\usepackage{physics}
\usepackage{enumerate}
\addtolength{\oddsidemargin}{-.5in}
	\addtolength{\evensidemargin}{-.5in}
	\addtolength{\textwidth}{1in}

	\addtolength{\topmargin}{-.5in}
	\addtolength{\textheight}{1in}
\begin{document}
\title{Tight-Binding Approximation}
\author{John Stanco}
\maketitle

\section*{Intro}

The tight–binding approximation is a method of calculating energy levels of a solid material.\\\\
To begin, we attempt to describe the material as an array of atoms in real space, called a lattice.  A lattice is simply a repetition of its constituent pieces, called unit cells.  The lattice in $R^n$ is thus generated by translating a unit cell in real space by any vector in the set of characteristic lattice vectors $\{a_1,...a_n\}$  (which can be thought of as translation operations). \\\\ 
Consequently, the geometry of a lattice will exhibit a translation symmetry, such that the lattice will be identical after translating by some vector $R = m_1a_1 + ... + m_na_n$.  The lattice vectors can thus form a basis in a subspace of $R^n$.\\\\
Technically speaking, the lattice can be characterized by any unit cell, as long as it can be associated with some set of vectors that can be used to construct the lattice.  However, it is common practice to describe a lattice via a primitive unit cell.  This is the unit cell containing the fewest atoms that can still be translated to generate the entire lattice.  \\\\
\\\\
Translational symmetry has a direct implication on the physical properties of the lattice.  First, define a momentum operator:
\begin{align*}
	\hat{p}\psi(r) &= -i\hbar\grad\psi(r)
\end{align*}
This acts on a wave function for an electron that is bound within the lattice.  Next, define a translation operator:
\begin{align*}
	T_a\psi(r) &= \lambda_a\psi(r - a)
\end{align*}
it allows one to model the potential due to each atomic orbital as a periodic potential, exhibiting the property $V(r) = V(r - R)$.  This will result in electron wave-functions exhibiting the same translational symmetry:
\begin{align*}
	\psi(r) &= \psi(r - R)
\end{align*}
We can see that both of these operators commute, because $\hat T_a\hat p\psi(r) = \hat p\hat T_a\psi(r)$., which implies that eigenstates of the momentum operator are also eigenstates of the translation operator.  This has a profound implication, as it states that the momentum of an electron within the lattice is completely translationally invariant!  If we define a position operator:
\begin{align*}
	\hat{r}\psi = r\psi(r)
\end{align*}
Examining the commutator:
\begin{align*}
	\hat{r}\hat T_a\psi(r) &= \hat r\lambda_a\psi(r - a) = (r - a)\lambda_a\psi(r - a)\\
	\hat T_a \hat{r} \psi(r) &= r\hat T_a \psi(r) = r\lambda_a\psi(r - a)
\end{align*}
Thus:
\begin{align*}
	[\hat r, \hat T_a] = \hat{r}\hat T_a - \hat T_a \hat{r} = -a\lambda_a
\end{align*}
Thus, the position operator does not commute with the translation operator, whereas the momentum operator does.  Using the fact that the momentum operator commutes with the Hamiltonian, we are able to express eigenstates of the Hamiltonian as eigenstates of the translation operator, denoted $\psi_k$.\\\\
These eigenstates must satisfy the property that $\braket{\psi_k} = 1$.  From the aforementioned definition of $T_a$, it follows that:
\begin{align*}
	|\lambda_a| = 1 \to \lambda_a = e^{i\phi a}
\end{align*}
For some arbitrary $\phi$.  This property is satisfied by the following set of basis states:
\begin{align*}
	\psi_k(r) = {1\over\sqrt{\Omega}}u_k(r)e^{ik\cdot r}
\end{align*}
Where $u_k(r)$ = $u_k(r - a)$, known as a Bloch function, has the same periodicity as the lattice itself.  As was mentioned before, the Hamiltonian will commute with the translation operator, where:
\begin{align*}
	\hat H = {\hat p^2\over 2m} + V(\hat r)
\end{align*}
This assertion relies on another important aspect of symmetry:  The potential, which exists due to the placement of the atoms, shares the same periodicity as the lattice itself, and thus $V(r) = V(r - a)$.  As a result, $\hat H$ and $\hat T_a$ share eigenstates, satisfying the Schrodinger equation:
\begin{align*}
\hat H\psi_{n,k}(r) = \epsilon_{n,k}\psi_{n,k}(r)
\end{align*}
\\\\
\section*{Solving $\hat H$}
There are a number of ways to solve this Hamiltonian
\section*{class 'TightBinding'}
Class that stores attributes of lattice (lattice vectors, etc), as well as all Tightbinding hopping energies.\\
Contains methods for calculating band structure, locating potential Weyl Nodes, and plotting the band Gap along planes
\section*{method 'tbBands'}

\section*{method 'findWeyl'}	

\section*{method 'plotGap'}

\section*{Weyl Nodes}
//From lattice to k-space lattice to real space to tight binding.
//Want to go from k-space to real space

\end{document}
